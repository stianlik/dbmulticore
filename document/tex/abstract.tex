\begin{abstract} 
The focus in this report is investigating efficient database
algorithms for modern multi-core processors in main memory
environments. It starts by introducing methods and algorithms used in
traditional database managements systems, one of the algorithms
described is the relatively new threshold algorithm for top-$k$ select
queries. Further, the multi-core landscape is explored in terms of
typical memory structures, cache hierarchies, frameworks, and common
optimization techniques. A selected set of memory based algorithms are
described, and the threshold algorithm is parallelized using
partitioning techniques similar to the ones used in traditional
parallel databases.

The parallel threshold algorithm is implemented and evaluated using a
set custom benchmarks to determine the effects of changing input size
and distribution, partition size, number of rows requested, and number
of threads used. Tests reveal that the algorithm achieve linear
speedup for a big uniform data set using two threads. However, the
relative speedup shows a gradual decrease as the number of threads are
increased, resulting in a speedup of only four when twelve threads are
in use. This decrease in relative speedup is not unexpected, it is
attributed the algorithms need for frequent synchronization, and a data
structure using coarse grained locking. 

Based on these results, it is evident that multi-core processors have
potential for increasing performance even in algorithms that require a
decent amount of synchronization. It is suggested that database
management systems should use heuristics to determine whether
multi-core algorithms will provide improved performance compared to
sequential versions in the process of query optimization. Continued
research in this area is likely to provide useful results, it is
essential that future database management systems utilize the
increased compute power provided by multi-core processors.
\end{abstract}
