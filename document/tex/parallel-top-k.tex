\section{Parallel top-k select}
\label{sec:parallel-top-k}

To utilize multiple cores for top-k, it is possible to run parts of TA
in parallel. This can achieve a certain degree of parallelism, but TA
is a synchronization heavy algorithm. This section will gradually
increase the degree of parallelism, while attempting to keep
synchronization costs low.

\subsection{Vertical input partitioning}

If the scoring function $f(\ldots)$ is in the form $ax + by + cy +
\ldots$, terms can be calculated in parallel. With expensive terms
this may have a impact on performance. Using as many threads as
attributes, costs are reduced by a factor of $n$, where $n$ is the
number of attributes. This can be accomplished using one master thread
to execute the algorithm and a pool of worker threads calculating the
terms on demand each time $f(\ldots)$ is called.

A shortcoming of this approach is that number of threads are limited
by number of attributes. Only the vertical potential for Parallelism
is exploited. In addition, worker threads can end up periodically
waiting for the master thread. This technique can be used in
combination with any of the following algorithms.

\subsection{Horizontal input partitioning}

Another way parallelize TA is to calculate one row per thread in a
round robin fashion. This allows for a greater degree of parallelism,
but requires some synchronization, see algorithm
\ref{alg:ta-parallel-basic}. 

\subsubsection{Algorithm description}

The function processRow contains lines 3-9 from the sequential
algorithm, in addition to some synchronization logic. When values are
written to R, a write lock is required. Checking if a key has been
processed and adding it to the set of processed keys is an atomic
operation. This may result in threads waiting to access collections.

\begin{algorithm}[H]
	\caption{HIP-TA}
	\label{alg:ta-parallel-basic}
	\begin{algorithmic}[1]
		\REQUIRE $
			L \leftarrow Collection\ of\ sorted\ lists,
			f \leftarrow Scoring\ function,
			k \leftarrow Requested\ number\ of\ tuples$
		\ENSURE $R = Sorted\ set\ of\ k\ tuples\ with\ highets\ score$

		\STATE $S \leftarrow \{\}$
		\COMMENT{All keys seen so far}
		\WHILE {$relation\ has\ more\ rows$}
			\FOR {i = 0\ to\ threadcount}
				\STATE $spawn\ processRow(L, f, S, R)$
			\ENDFOR
			\STATE $sync$
			\STATE $t \leftarrow f(\overline{p_0}, \ldots, \overline{p_n})$
			\IF {$\Big|\{r \in R | r.score \ge t\}\Big| \ge k$}
				\RETURN R
			\ENDIF
		\ENDWHILE
		\RETURN R
	\end{algorithmic}
\end{algorithm}

\subsubsection{Complexity analysis}

This is essentially the same algorithm as sequential TA. It use $processRow$
to do some of the operations in parallel so that number of iterations
in the outer loop (starting at line 2) is reduced by a factor $t$
(threadcount). In the ideal case, synchronization costs are trivial
and we get the following run-time:

$T(n,m,k,t) = O(\frac{mn^2}{t} + \frac{mnk}{t})$

The second term $\frac{mnk}{t}$ is optimistic if the data structure
and/or method of adding results remains equal to the sequential
version. When adding results, the entire list must be locked, forcing
this operation has to be run sequentially. This leads to a runtime of:

$T(n,m,k,t) = O(\frac{mn^2}{t} + mnk)$

In addition there are synchronization costs in each iteration, locks
need to be kept in memory, and checked before list is accessed. One
alternative is to use separate data structures for each thread, but
these need to be merged to generate the final result. Using a blocked
data structure may be a better option, allowing threads to lock
smaller parts of the result set in some cases. Blocking the result set
this way allows the algorithm to add some tuples in parallel.

\begin{comment}

\subsection{Horizontal input- and output partitioning}

TODO Rotete beskrivelse. Endre til spawn/sync-syntaks. Tenk på bedre
algoritme, tror ikke denne er helt ideell.

One way to reduce the need for synchronization is to delay adding
results to the final result set, and give each thread it's private
collection for seen objects. This way each thread can calculate a
predetermined number of rows before any synchronization is needed.
Some scores may be calculated multiple times since the algorithm does
not use a central set of seen elements. The impact of this depends
heavily on the input distribution, and on the cost of score
calculation. After all threads have calculated a predetermined number
of scores, a threshold value need to be calculated and results must be
combined. The algorithm choose the smallest threshold value from each
of the locally calculated results. Next, each partition is merged into
R (values below threshold is thrown away). The merge operation can be
expensive. See algorithm \ref{alg:splitmerge}.

\begin{description}
	\item[PF] Parallel framework
\end{description}

%\begin{algorithm}[H]
	%\caption{Naive parallel TA}
	%\label{alg:parallel-ta-1}
	%\begin{algorithmic}
		%\REQUIRE $
			%L \leftarrow Collection\ of\ sorted\ lists,
			%f \leftarrow Scoring\ function,
			%k \leftarrow Requested\ number\ of\ tuples$
		%\ENSURE $R = Sorted\ set\ of\ k\ tuples\ with\ highets\ score$
%
		%\STATE $n \leftarrow Number\ of\ scoring\ attributes$
		%\STATE $m \leftarrow Number\ of\ rows\ in\ relation$
		%\STATE $P = Values\ for\ last\ seen\ row, initially\ empty$
		%\COMMENT {one set per thread}
		%\STATE $F = All\ keys\ seen\ so\ far, initially\ empty$ 
%
		%\FOR[in parallel]{$i \leftarrow 0\ to\ m$}
			%\FORALL {$L_i \in L$}
				%\STATE unseen = $\neg F.findpush(L_i.key)$
				%\COMMENT{atomic}
				%\IF {unseen}
					%\STATE $score \leftarrow f(L_0.value, \ldots, L_n.value)$
					%\STATE $R.push\_if\_not\_worst(key,score)$
					%\COMMENT{atomic}
				%\ENDIF
				%\STATE $P[thread\_id]_i = L_i.value$
			%\ENDFOR
			%\STATE $T[thread\_id] = f(P_0, \ldots, P_n)$
			%\IF[atomic]{$T[thread\_id] < t$}
				%\STATE $t = T[thread\_id]$
			%\ENDIF
			%\STATE $barrier()$
			%\IF {$|\{r \in R | r.score \ge t\}| \ge k$}
				%\RETURN R
			%\ENDIF
		%\ENDFOR
		%\RETURN R
	%\end{algorithmic}
%\end{algorithm}

\begin{algorithm}[H]
	\label{alg:splitmerge}
	\caption{HIOP-TA}
	\begin{algorithmic}[1]
		\REQUIRE $
			L \leftarrow Collection\ of\ sorted\ lists,
			f \leftarrow Scoring\ function,
			k \leftarrow Requested\ number\ of\ tuples$
		\ENSURE $R \leftarrow Sorted\ set\ of\ k\ tuples\ with\ highets\ score$

		\STATE $t \leftarrow \infty$
		\COMMENT{threshold}
		\STATE $partsize \leftarrow 1$
		\COMMENT{each thread calculates one row per round}
		\STATE $startrow \leftarrow 0$

		\WHILE{$relation\ has\ more\ rows$}
			\STATE PF.begin(PARALLEL,thread\_count)
			\STATE $F_{threadid} = \{\}$
			\STATE $j \leftarrow startrow + thread\_id * partsize$ 
			\FOR{$i \leftarrow j\ to\ j + partsize$}
				\FORALL{$L_i \in L$}
					\IF{$key \notin F_{thread\_id}$}
						\STATE $score \leftarrow f(p_0,\ldots,p_n)$
						\STATE $F_{thread\_id} \leftarrow F_{thread\_id} 
						\cup \{(key, score)\}$
					\ENDIF
				\ENDFOR
			\ENDFOR
			\STATE $T_{thread\_id} \leftarrow f(\overline{p_0},\ldots,\overline{p_n})$
			\STATE PF.barrier()
			\STATE PF.begin(ATOMIC)
			\IF{$T_{thread\_id} < t$}
				\STATE $t \leftarrow T_{thread\_id}$
			\ENDIF
			\STATE PF.end(ATOMIC)
			\STATE $remove\ tuples\ scored \leq t\ from\ each\ list\ F_{threadid}$
			\STATE PF.end(PARALLEL)
			\STATE $ParallelMerge(F,R)$
			\IF {$|R| \ge k$}
				\RETURN R
			\ELSE
				\STATE $startrow \leftarrow startrow + thread\_count * partsize$
			\ENDIF
		\ENDWHILE
	\end{algorithmic}
\end{algorithm}

\begin{algorithm}[H]
	\label{alg:merge}
	\caption{ParallelMerge}
	\begin{algorithmic}
		\REQUIRE $
			L \leftarrow Collection\ of\ sorted\ lists,
			R \leftarrow Result\ array$
		\ENSURE $R \leftarrow L\ merged\ into\ one\ sorted\ list$
	
		\WHILE{$L.length \neq 1$}
			\STATE PF.begin(PARALLEL, $\lfloor \frac{L.length}{2} \rfloor$)
			\FOR{$i = 0\ to\ L.length\ increasing\ with\ 2\ each\ iteration$}
				\STATE $sequential\_merge((L_i, L_{i+1}), Tmp_{i/2})$
			\ENDFOR
			\STATE PF.end(PARALLEL)
			\STATE $L = Tmp$
		\ENDWHILE

		\STATE $R = L_1$

	\end{algorithmic}
\end{algorithm}

\subsection{Complexity analysis}

% n: number of scoring attributes
% m: number of tuples processed
% k: requested number of tuples
% p: partition size
% t: thread count

\begin{table}[H]
	\centering
	\begin{tabular}{|l|l|p{10cm}|}
		\hline
		Lines & Cost & Comment \\ \hline
		1-3	& $O(1)$ & \\
		4-5 & $O(\frac{m}{p})$ & \\
		6-7, 16-22, 24& $O(\frac{m}{pt})$ & \\
		8 & $O(\frac{m}{t})$ & \\
		9-10 & $O(\frac{mn}{t})$ & \\
		11 & $O(\frac{mn^2}{t})$ & \\
		12 & $O(\frac{lg(p)m}{t} + \frac{mp}{2t})$ & Binary search + moving values to keep list sorted \\
		23 & $O(\frac{m}{pt}*\lg(p))$ & Binary search. More expensive if
		many elements must be moved. Linked list and sequential search 
		may be beneficial \\
		25 & $O(2mt)$ & Depends om merge method, assuming 
		ParallelMerge is used and $t \rightarrow \infty$ merge will
		use $O(2tp)$, for lower values of $t$ it will be faster. \\
		26-30 & $O(\frac{m}{pt})$ \\
		\hline
	\end{tabular}
	\caption{HIOP-TA complexity analysis (run-time)}
\end{table}

$T(m,n,t,p)) = O(mt + \frac{mn^2}{t} + \frac{mp}{t} + \frac{m}{p} )$

\end{comment}
